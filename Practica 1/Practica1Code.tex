\documentclass[12pt]{article}
\usepackage[a4paper, total={6.1in, 10in}]{geometry} % Tipo de documento
\usepackage[utf8]{inputenc}   % Codificación de caracteres
\usepackage[T1]{fontenc}      % Codificación de fuentes
\usepackage{amsmath, amssymb} % Simbología matemática
\usepackage{graphicx}         % Insertar imágenes
\usepackage{hyperref}         % Hipervínculos
\usepackage{multicol}         % Multiples columnas}
\usepackage{xcolor}           % Colores
\usepackage{enumitem}         % Items romanos

% Información del documento
\pagecolor[RGB]{255, 255, 255}
\title{Practica 1: Lenguajes}
\author{Philips}
\date{1er Cuatrimestre 2025} % Fecha

\begin{document}

\maketitle % Genera el título

\section*{Ejercicio 1}
Sea $\Sigma$ = \{\textit{a, b}\} un alfabeto. Hallar:
\\
\\
\centerline{$\Sigma^0,\ \ \ \Sigma^1,\ \ \ \Sigma^2,\ \ \ \Sigma^*,\ \ \ \Sigma^+,\ \ \ |\Sigma|,\ \ \ |\Sigma^0|$}

\begin{itemize}
    \item $\Sigma^0 = \{\lambda\}$
    \item $\Sigma^1 = \{a,b\}$
    \item $\Sigma^2 = \{aa,ab,ba,bb\}$
    \item $\Sigma^* = \bigcup\limits_{i\geq0}\Sigma^i=\{\lambda,a,b,aa,ab,ba,bb,...\}$
    \item $\Sigma^+ = \bigcup\limits_{i\geq0}\Sigma^i=\{a,b,aa,ab,ba,bb,...\}$
    \item $|\Sigma| = 2$
    \item $|\Sigma^0| = 1$
\end{itemize}

\section*{Ejercicio 2}
Decidir si, dado $\Sigma$ = \{\textit{a, b}\}, vale:
\\
\\
\centerline{\(\lambda \in \Sigma, \quad \lambda \subseteq \Sigma, \quad \lambda \in \Sigma^+, \quad \lambda \in \Sigma^*, \quad \Sigma^0 = \lambda, \quad \Sigma^0 = \{\lambda\}\)}

\begin{itemize}
    \item  $\lambda \in \Sigma \equiv \textcolor{red}{False}$
    \item  $\lambda \subseteq \Sigma \equiv \textcolor{red}{False}$
    \item  $\lambda \in \Sigma^+ \equiv \textcolor{red}{False}$
    \item  $\lambda \in \Sigma^* \equiv \textcolor{blue}{True}$
    \item  $\Sigma^0 = \lambda \equiv \textcolor{red}{False}$
    \item  $\Sigma^0 = \{\lambda\} \equiv \textcolor{blue}{True}$
\end{itemize}

\section*{Ejercicio 3}
Sea \textit{$\alpha = abb$} una cadena. Calcular:
\\
\\
\centerline{$\alpha^0,\ \ \ \alpha^1,\ \ \ \alpha^2,\ \ \ \alpha^3,\ \ \ \prod_{k=0,...,3}{\alpha^k = \alpha^0.\alpha^1.\alpha^2.\alpha^3},\ \ \ \alpha^r$}

\begin{itemize}
    \item $\alpha^0 = \lambda$
    \item $\alpha^1 = \textit{abb}$
    \item $\alpha^2 = \textit{abb}{.}\textit{abb} = \textit{abbabb}$
    \item $\alpha^3 = \textit{abb}{.}\textit{abb}{.}\textit{abb} = \textit{abbabbabb}$
    \item $\prod_{k=0,...,3}{\alpha^k = \alpha^0.\alpha^1.\alpha^2.\alpha^3} = \lambda.\textit{abb}{.}\textit{abbabb}{.}\textit{abbabbabb}$
    \item $\alpha^r = (\textit{abb})^r = \textit{bba}$
\end{itemize}

\section*{Ejercicio 4}
Sean las cadenas \textit{$\alpha = abb$} y \textit{$\beta = acb$}. Calcular:
\\
\\
\centerline{\(\alpha\beta, \quad (\alpha\beta)^r, \quad (\beta)^r, \quad \beta^r\alpha^r, \quad \lambda\alpha , \quad \lambda\beta, \quad \alpha\lambda\beta, \quad \alpha^2\lambda^3\beta^2\)}

\begin{itemize}
    \item $\alpha\beta = abbacb$
    \item $(\alpha\beta)^r = bcabba$
    \item $(\beta)^r = bca$
    \item $\beta^r\alpha^r = bcabba$
    \item $\lambda\alpha = abb$
    \item $\lambda\beta = acb$
    \item $\alpha\lambda\beta = abbacb$
    \item $\alpha^2\lambda^3\beta^2 = abbabbacbacb$
\end{itemize}

\section*{Ejercicio 5}
Dado un alfabeto $\Sigma$, sean \textit{x},\textit{y} $\in \Sigma$ y $\alpha,\beta \in \Sigma^*$. Demostrar que:
\begin{enumerate}[label=\Roman*.]
    \item $|x.(y.\alpha)| = 2 + |\alpha|$
    \item $|\alpha^r|=|\alpha|$
    \item $|\alpha x \beta| = |x \alpha \beta|$
    \item $|\alpha . \alpha| = 2 |\alpha|$
    \item $(\alpha . \beta)^r = \beta^r . \alpha^r$
    \item $(\alpha^r)^r = \alpha$
    \item $(\alpha^r)^n = (\alpha^n)^r$
\end{enumerate}

\begin{enumerate}[label=\roman*.,font=\itshape]
    \item Quiero demostrar que $|x.(y.\alpha)| = 2 + |\alpha|$.
    \\
    \\
    {Como $x,y \in \Sigma$, usando la definición recursiva de longitud (||), los puedo "sacar" de la operación de la cadena y sumar un 1 cada vez que uso la ya mencionada definición.}
    \\
    \\
$|x\cdot(y\cdot\alpha)|\overset{\text{(||)}}{=} 1 + |y\cdot\alpha|\overset{\text{(||)}}{=} 2 + |\alpha|$
    \[
    \blacksquare
    \]
    \\ 
    \item Quiero demostrar que $|\alpha^r|=|\alpha|$. \\ 
    Pruebo por inducción estructural sobre $\alpha$: 
    \\
    \\
    \centerline{1. Caso base: Si $\alpha = \lambda$:}
    \\
    \centerline{$|\alpha|^r =|\lambda| \hspace{1cm} (\text{def. r} )$}
    \\
    \\
    \centerline{2. Caso inductivo: $\alpha = x.\beta$}
    \\
    \centerline{Suponemos que la propiedad vale para $\beta$}
    \\
    \\
    \centerline{$|x.\beta|^r = |\beta^r.x| \hspace{1cm} (\text{def. r})$}
    \centerline{$\quad = |\beta^r| + 1 \hspace{1cm} (\text{def. ||})$}
    \centerline{$\quad = |\beta| + 1 \hspace{1cm} (\text{def. HI})$}
    \centerline{$\quad = |x\cdot\beta| \hspace{1cm} (\text{def. ||})$}
    \centerline{$\quad = |\alpha| \hspace{1cm} (\text{def. $\alpha$})$}
    \[
    \blacksquare
    \]
    \item Quiero demostrar que $|\alpha x \beta| = |x \alpha \beta|$. \\
    Pruebo por inducción estructural sobre $\alpha$:
    \\
    \\
    \centerline{1. Caso base: Si $\alpha = \lambda$:}
    \\
    \centerline{$|\lambda x\beta| \overset{\text{$\lambda$}}{=} |x\lambda\beta| \overset{\text{$\alpha$}}{=} |x\alpha\beta|$}
    \\
    \\
    \centerline{2. Caso inductivo: $\alpha = y.\alpha'$}
    \\
    \centerline{Quiero ver si: $|y\alpha' x \beta| = |x y\alpha' \beta|$}
    \centerline{Suponemos que la propiedad vale para $\alpha'$}
    \\
    \\
    \centerline{$|(y\alpha')x\beta|\overset{\text{$||$}}{=} 1 + |\alpha'x\beta|$}
    \centerline{$\quad \overset{\text{$HI$}}{=} 1 + |x\alpha'\beta|$}
    \centerline{$\quad \overset{\text{||}}{=} 1 +1+ |\alpha'\beta|$}
    \centerline{$\quad \overset{\text{$y$}}{=} 1 + |y\alpha'\beta|$}
    \centerline{$\quad \overset{\text{$x$}}{=} |xy\alpha'\beta|$}
    \centerline{$\quad \overset{\text{$\alpha$}}{=} |x\alpha\beta|$}
    \[
    \blacksquare
    \]
    \item Quiero demostrar que $|\alpha . \alpha| = 2 |\alpha|$. \\
    Pruebo por inducción estructural sobre $\alpha$:
    \\
    \\
    \centerline{1. Caso base: Si $\alpha = \lambda$}
    \centerline{$|\lambda\cdot\lambda| \overset{\text{$\cdot$}}{=} |\lambda| \overset{\text{||}}{=} 0 \overset{\text{Int}}{=}2.0 \overset{\text{||}}{=}2.|\lambda|$}
    \\
    \\
    \centerline{2. Caso inductivo: $\alpha = x\beta$}
    \centerline{Quiero ver si: $|x\beta x\beta| = 2|x\beta|$}
    \centerline{Suponemos que la propiedad vale para $\beta$}
    \\
    \\
    \centerline{$|x\beta.x\beta| \overset{\text{||}}{=} 1 + |\beta x \beta|$}
    \centerline{$\quad \overset{\text{iii}}{=} 1 + |x\beta\beta|$}
    \centerline{$\quad \overset{\text{||}}{=} 1 + 1 + |\beta\beta|$}
    \centerline{$\quad \overset{\text{HI}}{=} 2(1 +|\beta|)$}
    \centerline{$\quad \overset{\text{||}}{=} 2(|x\beta|)$}
    \centerline{$\quad \overset{\text{$\alpha$}}{=} 2(|\alpha|)$}
    \[
    \blacksquare
    \]
    \item Quiero demostrar que $(\alpha . \beta)^r = \beta^r . \alpha^r$. \\
    Pruebo por inducción estructural sobre $\alpha$:
    \\
    \\
    \centerline{1. Caso base: Si $\alpha = \lambda$}
    \centerline{$(\lambda.\beta)^r \overset{\text{$\lambda$}}{=} \beta^r\overset{\text{$\lambda$}}{=} \beta^r.\lambda \overset{\text{$r$}}{=} \beta^r.\lambda^r$}
    \\
    \\
    \centerline{2. Caso inductivo: $\alpha = x\alpha'$}
    \centerline{Quiero ver si: $(x\alpha' . \beta)^r = \beta^r . (x\alpha')^r$}
    \centerline{Suponemos que la propiedad vale para $\beta$}
    \\
    \\
    ...    
    
    
\end{enumerate}

\section{Ejercicio 6}
Dar ejemplos de cadenas que pertenezcan a los siguientes lenguajes:
\begin{enumerate}[label=\Roman*.]
    \item $\mathcal{L} = \{a^nb^n \mid n\geq 0\}$
    \item $\mathcal{L} = \{a^nb^n \mid n\geq 1\}$
    \item $\mathcal{L} = \{a^nb^m \mid n\geq 1 \land m\geq 1\}$
    \item $\mathcal{L} = \{a^nb^m \mid n\geq 1 \land m\geq 0\}$
    \item $\mathcal{L} = \{a^n(ac)^p(bab)^q \mid n \leq 0 \land q = p +2 \land p \geq 1\}$
    \item $\mathcal{L} =  \{a,b\}^3 \cap \Lambda$
    \item $\mathcal{L} = \{ \alpha\alpha^r \mid \alpha \in \{a,b\}^+\}$
    \item $\mathcal{L} = \{ \alpha \in \{a,b\}^+ \mid \alpha = \alpha^+\}$
\end{enumerate}

\begin{enumerate}[label=\roman*.,font=\itshape]
    \item {$\mathcal{L} = \{a^nb^n \mid n\geq 0\}$
    \\
    \\
    {Todas las palabras de $\mathcal{L}$ tienen la misma cantidad de $a$ y $b$, y $\lambda \in \mathcal{L}$.}
    \\
    \\
    Ejemplos: \{$\lambda, ab, aabb, aaabbb$\}
    \item {$\mathcal{L} = \{a^nb^n \mid n\geq 1\}$
    \\
    \\    
    {Las palabras de $\mathcal{L}$ tienen, por lo menos, una $a$ y una $b$ ($\lambda \notin \mathcal{L}$), y la cantidad de apariciones de $a$ es igual a la cantidad de apariciones de $b$.}
    \\
    \\
    Ejemplos: \{$ab, aabb, aaabbb$\}
    \item {$\mathcal{L} = \{a^nb^m \mid n\geq 1 \land m\geq 1\}$
    \\
    \\
    {No es necesario que las cantidades de aparición de $a$ y $b$ sean iguales pero deben aparecer al menos una vez en cada palabra, y $\lambda \notin \mathcal{L}$.}
    \\
    \\
    Ejemplos: \{$ab, abb, aaaab, aaabbbb\}$}
    \item {$\mathcal{L} = \{a^nb^m \mid n\geq 1 \land m\geq 0\}$
    \\
    \\    
    {La cadena $b$ puede o no aparecer en la palabra. Por otro lado, la cadena $a$ siempre aparece, al menos, una vez en cada palabra de $\mathcal{L}$. Por último, $\lambda \notin \mathcal{L}$.}
    \\
    \\
    Ejemplos: \{$a, abb, aaaaabbbb$\}}
    \item {$\mathcal{L} = \{a^n(ac)^p(bab)^q \mid n \geq 0 \land q = p +2 \land p \geq 1\}$
    \\
    \\
    {Las palabras de $\mathcal{L}$ tienen, al menos, una aparición de $ac$, ya que se cumple que $p \geq 1$. Además, dado que $q=p+2$, esto implica que $q\geq3$, por lo que todas las cadenas tienen, al menos, tres apariciones consecutivas de la subcadena $bab$ (o más). Por otro lado, la cadena $a$ puede o no aparecer en la palabra, ya que $n \geq 0$. Por último, la cantidad de apariciones de $bab$ está directamente determinada por la cantidad de apariciones de $ac$, siguiendo la relación $q = p+2$. Esto significa que siempre habrá más apariciones de $bab$ que de $ac$ en cualquier cadena de $\mathcal{L}$.}
    \\
    \\
    Ejemplos: \{$aacbabbabbab, acacbabbabbabbab$\}}
    \item {$\mathcal{L} =\{\{a,b\}^3 \cap \Lambda\}$
    \\
    \\    
    {$\{a,b\}^3$ representa el conjunto de todas las cadenas de longitud $3$ formadas con los símbolos $a$ y $b$. Pero, como $\Lambda$ es el conjunto que contiene solamente la palabra vacía y todas las palabras que pertenecen a $\{a,b\}^3$ tienen longitud igual a $3$, la intersección entre ambos es vacía.}
    \\
    \\
    Ejemplos: \{$\emptyset$\}}
    \item {$\mathcal{L} = \{ \alpha\alpha^r \mid \alpha \in \{a,b\}^+\}$
    \\
    \\    
    {Todas las palabras de $\mathcal{L}$ se forman concatenando una cadena que pertenezca $\{a,b\}^+$ con su reversa. Todas estas palabras tienen longitud mayor o igual que 2, por lo que $\lambda \notin \mathcal{L}$.}
    \\
    \\
    Ejemplos: \{$aa, bb, baab$\}}
    \item {$\mathcal{L} = \{ \alpha \in \{a,b\}^+ \mid \alpha = \alpha^+\}$
    \\
    \\    
    {Todas las palabras de $\mathcal{L}$ son cadenas de la forma $\{a,b\}^+$ y cumplen que son iguales a su reversa. $\lambda \notin \mathcal{L}$ pues $\lambda \notin \{a,b\}^+$. PREGUNTAR}
    \\
    \\
    Ejemplos: \{$a, b, aa, bb, aba, bab$\}}}
    }
\end{enumerate}

\section*{Ejercicio 7}
Definir por comprensión los siguientes lenguajes:
\begin{enumerate}[label=\Roman*.]
    \item $\mathcal{L}_1 = \{ab, aabb, aaabbb,...\}$
    \item $\mathcal{L}_2 = \{aab, aaaabb, aaaaaabbb,...\}$
    \item $\mathcal{L}_3 = \{aaabccc, aaaabcccc, aaaaabccccc,...\}$
\end{enumerate}
\begin{enumerate}[label=\roman*.,font=\itshape]
    \item $\mathcal{L}_1 = \{a^nb^n \mid n\geq 1\}$
    \item $\mathcal{L}_2 = \{a^nb^m \mid n =m*2 \land m\geq1\}$
    \item $\mathcal{L}_3 = \{a^nbc^n \mid n\geq3\}$
\end{enumerate}

\section*{Ejercicio 8}
Dados $\mathcal{L}_1= \{a,bc\}, \mathcal{L}_2 = \{aaa,bc\}$, y siendo $\Lambda = \{\lambda\}$, calcular:
\begin{enumerate}[label=\Roman*.]
    \item $\mathcal{L}_1 \cup \mathcal{L}_2 = \{a,bc,aaa\}$
    \item $\mathcal{L}_1 \cap \mathcal{L}_2 = \{bc\}$
    \item $\mathcal{L}_1 \cdot \mathcal{L}_2 = \{aaaa,abc,bcaaa,bcbc\}$
    \item $\mathcal{L}_1 \cdot (\mathcal{L}_2)^0 = \{a,bc\}$
    \item $\mathcal{L}_1 \cdot (\mathcal{L}_2)^2 = \mathcal{L}_1 \cdot\{aaaaaa,aaabc,bcaaa,bcbc\} \\
    = \{aaaaaaa,aaaabc,abcaaa,abcbc,bcaaaaaa,bcaaabc,bcbcaaa,bcbcbc\}$
    \item $\mathcal{L}_1 \cdot (\mathcal{L}_2)^+ = \mathcal{L}_1 \cdot\{aaa,bc,aaaaaa,bcbc,aaabc,bcaaa,...\} = \{aaaa,abc,bcbc,abcbc,bcaaabc,...\}$
    \item $(\mathcal{L}_1 \cdot \mathcal{L}_2)^+ = \{aaaa,abc,bcaaa,bcbc,bca,aaabc,...\}$
    \item $(\mathcal{L}_1 \cdot \mathcal{L}_2)^* = \{\lambda,a,aaa,bc,aaaa,abc,bcaaa,bcbc,bca,aaabc,...\}$
    \item $\mathcal{L}_1 \cdot \Lambda \cdot \mathcal{L}_2 = \mathcal{L}_1 \cdot \mathcal{L}_2$
    \item $\mathcal{L}_1 \cdot \emptyset \cdot \mathcal{L}_2 = \emptyset$
    \item $(\mathcal{L}_1)^r = \{a^r,(bc)^r\} =\{a,cb\}$ PREGUNTAR
    \item $(\mathcal{L}_1 \cdot \mathcal{L}_2)^r = \{aaaa,abc,bcaaa,bcbc\}^r$ \\
    $ = \{(aaaa)^r,(abc)^r,(bcaaa)^r,(bcbc)^r\} = \{aaaa,cba,aaacb,cbcb\}$ PREGUNTAR
\end{enumerate}

\section*{Ejercicio 9}
Determinar el complemento de los siguientes lenguajes, considerando los alfabetos indicados en cada caso.
\begin{enumerate}[label=\Roman*.]
    \item $\mathcal{L} = \Lambda$ para $\Sigma = \{a, b\}$
    \item $\mathcal{L} = \{\lambda, a\}$ para $\Sigma = \{a\}$ y $\Sigma = \{a, b\}$
    \item $\mathcal{L} = \{b\alpha \mid \alpha \in \{a, b\}^*\}$ para $\Sigma = \{a, b\}$
    \item $\mathcal{L} = \{a^{2n} \mid n \geq 0\}$ para $\Sigma = \{a\}$ y $\Sigma = \{a, b\}$
    \item $\mathcal{L} = \{\alpha_1 b \alpha_2 \mid \alpha_1, \alpha_2 \in \{a, b\}^* \land |\alpha_1| > |\alpha_2|\}$ para $\Sigma = \{a, b\}$
\end{enumerate}
\begin{enumerate}[label=\roman*.,font=\itshape]
    \item $\mathcal{L}^c = \Sigma^*$
    \item $\Sigma^* \backslash \{\lambda,a\} = \{aa,aaa,aaaa,...\}$ \\
          $\Sigma^* \backslash \{\lambda, a\} =$ Todas las cadenas que se pueden formar con $a$ y $b$, sin $\lambda$ y la cadena "$a$".
    \item $\mathcal{L}^c= \{\lambda\} \cup \{a\alpha \mid \alpha \in \{a,b\}^*\}$
    \item $\{a^{2n +1} \mid n \geq 0\}$ Cadenas con cantidad impar de "$a$".\\
          $\{a^{2n +1} \mid n \geq 0 \} \cup \{\alpha \mid \alpha \in \{a,b\}^* \land |\alpha|_b \geq 1\}$ Cadenas con cantidad impar de "$a$" y cadenas que contienen al menos una "$b$".
    \item $\{\alpha_1b\alpha_2 \mid \alpha_1, \alpha_2 \in \{a, b\}^* \land |\alpha_1| \leq |\alpha_2|\} \cup \{\alpha \mid \alpha \in \{a,b\}^* \land |\alpha|_b = 0\}$ \\
    Cadenas de forma $\alpha_1b\alpha_2$ pero con $|\alpha_1| \leq |\alpha_2|$, y cadenas que no contienen la "$b$".
\end{enumerate}

\section*{Ejercicio 10}
Sea $\mathcal{L}$, $\mathcal{L}_1$, $\mathcal{L}_2$ lenguajes cualesquiera. Determinar si las siguientes afirmaciones son verdaderas o falsas. Si son verdaderas, demostrarlas. Si no, dar un contraejemplo.
\begin{enumerate}[label=\Roman*.]
    \item $\mathcal{L}^+ \subseteq \mathcal{L}^* \equiv \textcolor{blue}{True}$
    \item $\mathcal{L}^+ \subsetneq \mathcal{L}^* \equiv \textcolor{red}{False}$
    \item $\mathcal{L}^n \cdot \mathcal{L}^m = \mathcal{L}^{n+m} \hspace{0.1cm} \text{para todo } n,m \geq 0 \equiv \textcolor{blue}{True}$
    \item $\mathcal{L}^n \subseteq \mathcal{L}^{n+1} \hspace{0.1cm} \text{para todo } n \geq 0\equiv \textcolor{blue}{True}$
    \item $\mathcal{L}_1 \subseteq \mathcal{L}_2, n \geq 0 \Rightarrow (\mathcal{L}_1)^n \subseteq (\mathcal{L}_2)^n \equiv \textcolor{blue}{True}$
    \item $\mathcal{L}_1 \subseteq \mathcal{L}_2 \Rightarrow (\mathcal{L}_1)^* \subseteq (\mathcal{L}_2)^* \equiv \textcolor{blue}{True}$
    \item $(\mathcal{L}^*)^* = \mathcal{L}^* \equiv \textcolor{blue}{True}$
    \item $(\mathcal{L}^+)^+ = \mathcal{L}^* \equiv \textcolor{red}{False}$
    \item $(\mathcal{L}^+)^* = \mathcal{L}^* \equiv \textcolor{blue}{True}$
    \item $(\mathcal{L}_1 \cup \mathcal{L}_2)^* = (\mathcal{L}_1)^* \cup (\mathcal{L}_2)^* \equiv \textcolor{red}{False}$
    \item $(\mathcal{L}_1 \cap \mathcal{L}_2)^* = (\mathcal{L}_1)^* \cap (\mathcal{L}_2)^* \equiv \textcolor{red}{False}$
    \item $(\mathcal{L}^2)^* = \mathcal{L}^* \equiv \textcolor{red}{False}$
    \item $(\mathcal{L} \cup \mathcal{L}_2)^* = \mathcal{L}^* \equiv \textcolor{blue}{True}$
    \item $(\mathcal{L}^n)^r = (\mathcal{L}^r)^n \hspace{0.1cm} \text{para todo } n \geq 0$
    \item $(\mathcal{L}^*)^r = (\mathcal{L}^r)^* \equiv \textcolor{blue}{True}$
\end{enumerate}
\begin{enumerate}[label=\roman*.,font=\itshape]
    \item Por definición: \\ 
    $\mathcal{L}^+ = \bigcup_{n\geq1}\mathcal{L}^n \subseteq \bigcup_{n\geq1}\mathcal{L}^n \cup \mathcal{L}^0 = \bigcup_{n\geq0}\mathcal{L}^n = \mathcal{L}^*$
    \item Contraejemplo: \\
    Sea $\mathcal{L} = \{\lambda, a\}$, entonces $\mathcal{L}^+ \subseteq \mathcal{L}^*$ pero $\mathcal{L}^+ = \mathcal{L}^*$.
    \item $3$.
    \item $4$.
    \item Por inducción en $n$: \\
    $\bullet$ $\hspace{0.01cm}$ Caso base: si $n=0$, ${\mathcal{L}_1}^0 = {\mathcal{L}_2}^0 = \Lambda$. \\
    $\bullet$ $\hspace{0.01cm}$ Caso inductivo: $n = m +1$. \\
    Supongo que para $m \geq 0$ se cumple que: ${\mathcal{L}_1}^m \subseteq {\mathcal{L}_2}^m$. Y, quiero ver que se cumple para $n = m + 1$. \\
    Sea $\alpha \in {\mathcal{L}_1}^n$, esto quiere decir que $\alpha = \beta\cdot\gamma$ con $\beta \in \mathcal{L}_1$ y $\gamma \in {\mathcal{L}_1}^m$. \\
    Entonces, $\beta \in \mathcal{L}_2$ y $\gamma \in {\mathcal{L}_2}^m$, pues por $HI$ $\mathcal{L}_1 \subseteq \mathcal{L}_2$ y ${\mathcal{L}_1}^m \subseteq {\mathcal{L}_2}^m$. Luego, $\alpha \in {\mathcal{L}_2}^{m+1} = {\mathcal{L}_2}^{n}$.
    \item $6$.
    \item $7$.
    \item Contraejemplo: \\
    Sea $\mathcal{L} = \{a\} \to \lambda \notin (\mathcal{L}^+)^+$, pero $\lambda \in (\mathcal{L}^*)$. Por lo tanto, $(\mathcal{L}^+)^+ \neq (\mathcal{L}^*)$.
    \item $9$.
    \item Contraejemplo: \\
    Sea $\mathcal{L}_1 = \{a\},\mathcal{L}_2 = \{b\}, ab \in (\mathcal{L}_1\cup \mathcal{L}_2)^*$ pero $ab \notin \mathcal{L}_1^* \cup \mathcal{L}_2^*$.
    \item Contraejemplo: \\
    Sean $\mathcal{L}_1 = b$ y $\mathcal{L}_2 = bb$. \\
    $(\mathcal{L}_1 \cap \mathcal{L}_2)^* = \emptyset^* = \Lambda$. Pero, $(\mathcal{L}_1)^* \cap (\mathcal{L}_2)^* = \{\lambda, a,aa,aaa,...\} \cap \{\lambda, aa, aaaa,...\} \neq \Lambda$.
    \item Contraejemplo: \\
    Sea $\mathcal{L} = \{a\}, a \in \mathcal{L}^*$ pero $a \notin (\mathcal{L}_2)^*$.
    \item $13$.
    \item $14$.
    \item $15$.
\end{enumerate}

\section{Ejercicio 11}
Siendo: 
\begin{itemize}
    \item $Sub(\mathcal{L})$: subcadenas del lenguaje $\mathcal{L}$.
    \item $Ini(\mathcal{L})$: subcadenas iniciales (prefijos) del lenguaje $\mathcal{L}$.
    \item $Fin(\mathcal{L})$: subcadenas finales (sufijos) del lenguaje $\mathcal{L}$.
\end{itemize}
Demostrar que si $\mathcal{L}_1$ y $\mathcal{L}_2$ son lenguajes:
\begin{enumerate}[label = \Roman*.]
    \item $\text{Fin}(\text{Fin}(\mathcal{L}_1)) = \text{Fin}(\mathcal{L}_1)$.
    \item $\text{Sub}(\text{Sub}(\mathcal{L}_1)) = \text{Sub}(\mathcal{L}_1)$.
    \item $\text{Fin}(\mathcal{L}_1 \mathcal{L}_2) = \text{Fin}(\mathcal{L}_2) \cup \text{Fin}(\mathcal{L}_1) \mathcal{L}_2$.
    \item $\text{Ini}(\mathcal{L}_1 \cup \mathcal{L}_2) = \text{Ini}(\mathcal{L}_1) \cup \text{Ini}(\mathcal{L}_2)$.
    \item $\text{Fin}(\mathcal{L}_1 \cup \mathcal{L}_2) = \text{Fin}(\mathcal{L}_1) \cup \text{Fin}(\mathcal{L}_2)$.
    \item $\text{Ini}(\text{Sub}(\mathcal{L}_1)) = \text{Sub}(\text{Ini}(\mathcal{L}_1)) = \text{Fin}(\text{Sub}(\mathcal{L}_1)) = \text{Sub}(\text{Fin}(\mathcal{L}_1)) = \text{Sub}(\mathcal{L}_1)$.
\end{enumerate}
\begin{enumerate}[label=\roman*.,font=\itshape]
    \item $1$.
    \item $2$.
    \item $3$.
    \item $4$.
    \item $5$.
    \item $6$.
\end{enumerate}
\end{document}
