\documentclass[12pt]{article}
\usepackage[a4paper, total={6.1in, 10in}]{geometry} % Tipo de documento
\usepackage[utf8]{inputenc}   % Codificación de caracteres
\usepackage[T1]{fontenc}      % Codificación de fuentes
\usepackage{amsmath, amssymb} % Simbología matemática
\usepackage{graphicx}         % Insertar imágenes
\usepackage{hyperref}         % Hipervínculos
\usepackage{multicol}         % Multiples columnas}
\usepackage{xcolor}           % Colores
\usepackage{enumitem}         % Items romanos

\title{Practica 1 - Definiciones y demostraciones EXTRAS}
\author{Philips}
\date{1er Cuatrimestre 2025}

\begin{document}

\maketitle

\section{Definiciones sobre palabras y cadenas:}
\renewcommand\labelenumi{(\theenumi)}
\begin{enumerate}
    \item {Definición recursiva de longitud:}
    \\
    \\
    \centerline{$|\lambda| \overset{\text{(L0)}}{=} 0$}
    \\
    \\
    \centerline{$|x.\alpha| \overset{\text{(L1)}}{=} 1 + |\alpha|$}
    \item {Definición recursiva de la cantidad de apariciones:}
    \\
    \\
    \centerline{$|\lambda|_x \overset{\text{(a0)}}{=} 0$}
    \\
    \\
    \[
    |y \cdot \alpha|_x \overset{\text{(a1)}}{=}
    \begin{cases} 
    1 + |\alpha|_x & \text{si } y = x \\
    |\alpha|_x & \text{si } y \neq x
    \end{cases}
    \]
    \item {Definición recursiva de reversa:}
    \\
    \\
    \centerline{$\lambda^r \overset{\text{(r0)}}{=} \lambda$}
    \\
    \\
    \centerline{$(x\alpha)^r \overset{\text{(r1)}}{=} \alpha^r.x$}
    \item {Definición recursiva de potencia:}
    \\
    \\
    \centerline{$\alpha^0 \overset{\text{(p0)}}{=} \lambda$}
    \\
    \\
    \centerline{$\alpha^{n+1} \overset{\text{(p1)}}{=} \alpha\cdot\alpha^n$}
\end{enumerate}

\section{Definiciones sobre Lenguajes}
\renewcommand\labelenumi{(\theenumi)}
\begin{enumerate}
    \item {Complemento:}
    \\
    \\
    \centerline{$\mathcal{L}^c=\Sigma^*\backslash\mathcal{L}$}
    \item {Unión:}
    \\
    \\
    \centerline{Dados $\mathcal{L}_1,\mathcal{L}_2 \subseteq \Sigma^*, \hspace{0.2cm} \mathcal{L}_1\cup\mathcal{L}_2 = \{\alpha \in \Sigma^* \mid \alpha \in \mathcal{L}_1\lor \alpha \in \mathcal{L}_2\}$}
    \item {Intersección:}
    \\
    \\
    \centerline{Dados $\mathcal{L}_1,\mathcal{L}_2 \subseteq \Sigma^*, \hspace{0.2cm} \mathcal{L}_1\cap\mathcal{L}_2 = \{\alpha \in \Sigma^* \mid \alpha \in \mathcal{L}_1\land \alpha \in \mathcal{L}_2\}$}
    \item {Reverso:}
    \\
    \\
    \centerline{Dado $\mathcal{L} \subseteq \Sigma^*,\hspace{0.2cm} \mathcal{L}^r= \{\alpha^r \mid \alpha \in \mathcal{L}\}$}
    \item {Concatenación:}
    \\
    \\
    \centerline{Dados $\mathcal{L}_1,\mathcal{L}_2 \subseteq \Sigma^*,\hspace{0.2cm} \mathcal{L}_1\cdot\mathcal{L}_2= \{\alpha \cdot \beta \mid \alpha \in \mathcal{L}_1 \land \beta \in \mathcal{L}_2\}$}
    \item {Potencia:}
    \\
    \\
    \[
    \mathcal{L}^n =
    \begin{cases} 
    \Lambda & \text{si } n = 0 \\
    \mathcal{L}\cdot\mathcal{L}^{n-1} & \text{si } n > 0
    \end{cases}
    \]
    \item {Clausura de Kleene:}
    \\
    \\
    \centerline{$\mathcal{L}^*=  \bigcup_{n \geq 0} \mathcal{L}^{n}$}
    \item {Clausura positiva:}
    \\
    \\
    \centerline{$\mathcal{L}^+=  \bigcup_{n \geq 1} \mathcal{L}^{n}$}
    \item {Prefijos}
    \\
    Sea $\mathcal{}{L} \subseteq \Sigma^{*}$ un lenguaje:
    \\
    \\
    \centerline{$\text{Ini}(\mathcal{L}) = \{ \alpha \in \Sigma^* \mid \exists \beta \in \Sigma^* \text{ tal que } \alpha\beta \in \mathcal{L} \}$}
    \item {Sufijos}
    \\
    Sea $\mathcal{}{L} \subseteq \Sigma^{*}$ un lenguaje:
    \\
    \\
    \centerline{$\text{Fin}(\mathcal{L}) = \{ \alpha \in \Sigma^* \mid \exists \beta \in \Sigma^* \text{ tal que } \beta\alpha \in \mathcal{L} \}$}
    \item {Subcadenas}
    \\
    Sea $\mathcal{}{L} \subseteq \Sigma^{*}$ un lenguaje:
    \\
    \\
    \centerline{$\text{Sub}(\mathcal{L}) = \{ \alpha \in \Sigma^* \mid \exists \beta, \gamma \in \Sigma^* \text{ tal que } \beta\alpha\gamma \in \mathcal{L} \}$}
\end{enumerate}

\section{Demostraciones básicas:}
\renewcommand\labelenumi{(\theenumi)}
\begin{enumerate}
    \item {Propiedad: $|\alpha . \beta| = |\alpha|+|\beta|$}
    \\
    Demo por inducción estructural sobre $\alpha$:
    \\
    \\
    \centerline{1. Caso base: Si $\alpha = \lambda$:}
    \\
    \centerline{$|\lambda.\beta| = |\beta| = 0 + |\beta| = |\lambda| + |\beta|$}
    \\
    \\
    \centerline{2. Caso inductivo: $\alpha = x.\alpha'$}
    \\
    \centerline{Suponemos que la propiedad vale para $\alpha'$}
    \\
    \\
    \centerline{$|\alpha \cdot \beta| = |(x.\alpha').\beta| \hspace{1cm} (\text{def. } \alpha)$}
    \\
    \centerline{$= 1 + |\alpha' \cdot \beta | \hspace{1cm} (\text{def. } |\cdot|)$}
    \\
    \centerline{$= 1 + |\alpha'| + |\beta| \hspace{1cm} (HI)$}
    \\
    \centerline{$= |x \cdot \alpha'| + |\beta| \hspace{1cm} (\text{def. } |\cdot|)$}
    \\
    \centerline{$= |\alpha| + |\beta| \hspace{1cm} (\text{def. } \alpha)$}
    \item{lol}
\end{enumerate}

\end{document}
